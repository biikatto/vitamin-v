\hyperlink{class_rt_audio}{Rt\+Audio} makes restrained use of C++ exceptions. That is, exceptions are thrown only when system errors occur that prevent further class operation or when the user makes invalid function calls. In other cases, a warning message may be displayed and an appropriate value is returned. For example, if a system error occurs when processing the \hyperlink{class_rt_audio_a747ce2d73803641bbb66d6e78092aa1a}{Rt\+Audio\+::get\+Device\+Count()} function, the return value is zero. In such a case, the user cannot expect to make use of most other \hyperlink{class_rt_audio}{Rt\+Audio} functions because no devices are available (and thus a stream cannot be opened). A client can call the function \hyperlink{class_rt_audio_af0752ee51cce3dd90a3bd009f9fdbe77}{Rt\+Audio\+::show\+Warnings()} with a boolean argument to enable or disable the printing of warning messages to {\ttfamily stderr}. By default, warning messages are displayed. There is a protected \hyperlink{class_rt_audio}{Rt\+Audio} method, error(), that can be modified to globally control how these messages are handled and reported. 