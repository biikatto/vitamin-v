\hyperlink{class_rt_audio}{Rt\+Audio} is designed to provide a common A\+PI across the various supported operating systems and audio libraries. Despite that, some issues should be mentioned with regard to each.\hypertarget{apinotes_linux}{}\section{Linux\+:}\label{apinotes_linux}
\hyperlink{class_rt_audio}{Rt\+Audio} for Linux was developed under Redhat distributions 7.\+0 -\/ Fedora. Four different audio A\+P\+Is are supported on Linux platforms\+: \href{http://www.opensound.com/oss.html}{\tt O\+SS} (versions $>$= 4.\+0), \href{http://www.alsa-project.org/}{\tt A\+L\+SA}, \href{http://jackit.sourceforge.net/}{\tt Jack}, and \href{http://www.freedesktop.org/wiki/Software/PulseAudio}{\tt Pulse\+Audio}. \hyperlink{class_note}{Note} that \hyperlink{class_rt_audio}{Rt\+Audio} now only supports the newer version 4.\+0 O\+SS A\+PI. The A\+L\+SA A\+PI is now part of the Linux kernel and offers significantly better functionality than the O\+SS A\+PI. \hyperlink{class_rt_audio}{Rt\+Audio} provides support for the 1.\+0 and higher versions of A\+L\+SA. Jack is a low-\/latency audio server written primarily for the G\+N\+U/\+Linux operating system. It can connect a number of different applications to an audio device, as well as allow them to share audio between themselves. Input/output latency on the order of 15 milliseconds can typically be achieved using any of the Linux A\+P\+Is by fine-\/tuning the \hyperlink{class_rt_audio}{Rt\+Audio} buffer parameters (without kernel modifications). Latencies on the order of 5 milliseconds or less can be achieved using a low-\/latency kernel patch and increasing F\+I\+FO scheduling priority. The pthread library, which is used for callback functionality, is a standard component of all Linux distributions.

The A\+L\+SA library includes O\+SS emulation support. That means that you can run programs compiled for the O\+SS A\+PI even when using the A\+L\+SA drivers and library. It should be noted however that O\+SS emulation under A\+L\+SA is not perfect. Specifically, channel number queries seem to consistently produce invalid results. While O\+SS emulation is successful for the majority of \hyperlink{class_rt_audio}{Rt\+Audio} tests, it is recommended that the native A\+L\+SA implementation of \hyperlink{class_rt_audio}{Rt\+Audio} be used on systems which have A\+L\+SA drivers installed.

The A\+L\+SA implementation of \hyperlink{class_rt_audio}{Rt\+Audio} makes no use of the A\+L\+SA \char`\"{}plug\char`\"{} interface. All necessary data format conversions, channel compensation, de-\/interleaving, and byte-\/swapping is handled by internal \hyperlink{class_rt_audio}{Rt\+Audio} routines.\hypertarget{apinotes_macosx}{}\section{Macintosh O\+S-\/\+X (\+Core\+Audio and Jack)\+:}\label{apinotes_macosx}
The Apple Core\+Audio A\+PI is designed to use a separate callback procedure for each of its audio devices. A single \hyperlink{class_rt_audio}{Rt\+Audio} duplex stream using two different devices is supported, though it cannot be guaranteed to always behave correctly because we cannot synchronize these two callbacks. The {\itshape number\+Of\+Buffers} parameter to the \hyperlink{class_rt_audio_a6907539d2527775df778ebce32ef1e3b}{Rt\+Audio\+::open\+Stream()} function has no affect in this implementation.

It is not possible to have multiple instances of \hyperlink{class_rt_audio}{Rt\+Audio} accessing the same Core\+Audio device.

The \hyperlink{class_rt_audio}{Rt\+Audio} Jack support can be compiled on Macintosh O\+S-\/X systems, as well as in Linux.\hypertarget{apinotes_windowsds}{}\section{Windows (\+Direct\+Sound)\+:}\label{apinotes_windowsds}
The {\ttfamily configure} script provides support for the Min\+GW compiler. Direct\+Sound support is specified with the \char`\"{}-\/-\/with-\/ds\char`\"{} flag.

In order to compile \hyperlink{class_rt_audio}{Rt\+Audio} under Windows for the Direct\+Sound A\+PI, you must have the header and source files for Direct\+Sound version 5.\+0 or higher. As far as I know, there is no Direct\+Sound\+Capture support for Windows NT. Audio output latency with Direct\+Sound can be reasonably good, especially since \hyperlink{class_rt_audio}{Rt\+Audio} version 3.\+0.\+2. Input audio latency still tends to be bad but better since version 3.\+0.\+2. \hyperlink{class_rt_audio}{Rt\+Audio} was originally developed with Visual C++ version 6.\+0 but has been tested with .N\+ET.

The Direct\+Sound version of \hyperlink{class_rt_audio}{Rt\+Audio} can be compiled with or without the U\+N\+I\+C\+O\+DE preprocessor definition.\hypertarget{apinotes_windowsasio}{}\section{Windows (\+A\+S\+I\+O)\+:}\label{apinotes_windowsasio}
A\+S\+IO support using Min\+GW and the {\ttfamily configure} script is specified with the \char`\"{}-\/-\/with-\/asio\char`\"{} flag.

The Steinberg A\+S\+IO audio A\+PI allows only a single device driver to be loaded and accessed at a time. A\+S\+IO device drivers must be supplied by audio hardware manufacturers, though A\+S\+IO emulation is possible on top of systems with Direct\+Sound drivers. The {\itshape number\+Of\+Buffers} parameter to the \hyperlink{class_rt_audio_a6907539d2527775df778ebce32ef1e3b}{Rt\+Audio\+::open\+Stream()} function has no affect in this implementation.

A number of A\+S\+IO source and header files are required for use with \hyperlink{class_rt_audio}{Rt\+Audio}. Specifically, an \hyperlink{class_rt_audio}{Rt\+Audio} project must include the following files\+: {\ttfamily \hyperlink{asio_8h_source}{asio.\+h},cpp; \hyperlink{asiodrivers_8h_source}{asiodrivers.\+h},cpp; \hyperlink{asiolist_8h_source}{asiolist.\+h},cpp; \hyperlink{asiodrvr_8h_source}{asiodrvr.\+h}; \hyperlink{asiosys_8h_source}{asiosys.\+h}; \hyperlink{ginclude_8h_source}{ginclude.\+h}; \hyperlink{iasiodrv_8h_source}{iasiodrv.\+h}; \hyperlink{iasiothiscallresolver_8h_source}{iasiothiscallresolver.\+h},cpp}. The Visual C++ projects found in {\ttfamily /tests/\+Windows/} compile both A\+S\+IO and Direct\+Sound support.

The Steinberg provided {\ttfamily asiolist} class does not compile when the preprocessor definition U\+N\+I\+C\+O\+DE is defined. \hyperlink{class_note}{Note} that this could be an issue when using \hyperlink{class_rt_audio}{Rt\+Audio} with Qt, though Qt programs appear to compile without the U\+N\+I\+C\+O\+DE definition (try {\ttfamily D\+E\+F\+I\+N\+ES -\/= U\+N\+I\+C\+O\+DE} in your .pro file). \hyperlink{class_rt_audio}{Rt\+Audio} with A\+S\+IO support has been tested using the Min\+GW compiler under Windows XP, as well as in the Visual Studio environment. 